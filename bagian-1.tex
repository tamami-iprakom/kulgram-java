\documentclass{beamer}
\usetheme{Boadilla}

% - for list code
\usepackage{color}
\usepackage{listings}
% - for list code

% list code program
\definecolor{codegreen}{rgb}{0,0.6,0}
\definecolor{codegray}{rgb}{0.5,0.5,0.5}
\definecolor{codepurple}{rgb}{0.58,0,0.82}
\definecolor{backcolor}{rgb}{0.95,0.95,0.92}

\lstdefinestyle{mystyle}{
  backgroundcolor=\color{backcolor},
  commentstyle=\color{codegreen},
  keywordstyle=\color{magenta},
  stringstyle=\color{codepurple},
  basicstyle=\footnotesize,
  breakatwhitespace=false,
  breaklines=true,
  captionpos=b,
  keepspaces=true,
  numbers=left,
  numbersep=5pt,
  showspaces=false,
  showstringspaces=false,
  showtabs=false,
  tabsize=2
}

\lstset{style=mystyle}

\def\beamer@verbatimreadframe{%
  \begingroup%
  \let\do\beamer@makeinnocent\dospecials%
  \count@=127%
  \@whilenum\count@<255 \do{%
    \advance\count@ by 1%
    \catcode\count@=11%
  }%
  \beamer@makeinnocent\^^L% and whatever other special cases
  \beamer@makeinnocent\^^I% <-- PATCH: allows tabs to be written to temp file
  \endlinechar`\^^M \catcode`\^^M=12%
  \@ifnextchar\bgroup{\afterassignment\beamer@specialprocessframefirstline\let\beamer@temp=}{\beamer@processframefirstline}}%

% end list code program

\title{Java}
\subtitle{Bagian 1}
\author{tamami}
\institute{BPPKAD Kab. Brebes}
\date{\today}

\begin{document}

\begin{frame}
\titlepage
\end{frame}

\section{Perkenalan}
\subsection{Apa sih?}

%%% - frame 1

\begin{frame}
\frametitle{Java, apa?}

\begin{itemize}
	\item Java itu, compiler dan interpreter
	\item Java itu, OOP
\end{itemize}
\end{frame}


%%% - frame 2

\begin{frame}
\frametitle{Pasang yu}

\begin{itemize}
	\item JDK??
	\item JRE??
\end{itemize}
\end{frame}


%%% - frame 3

\begin{frame}[fragile]
\frametitle{Coba yu}
\begin{lstlisting}
public class Coba {
  public static void main(String args[]) {
    System.out.println("Hai, apa kabar?");
  }
}
\end{lstlisting}
\end{frame}


%%% - frame 4

\begin{frame}
\frametitle{Ketentuan}
Beberapa adat yang dilakukan :

\begin{itemize}
	\item CamelCase
	\item Nama File = Nama Kelas
	\item Nama Paket = Nama Direktori dalam \textit{Project}
\end{itemize}

\end{frame}

\end{document}